\documentclass[12pt, oneside, titlepage]{article}   	% use "amsart" instead of "article" for AMSLaTeX format

\usepackage{graphicx}
\graphicspath{ {\string} }
\usepackage{subcaption}

%%%%%%%%%%%%%%%%%%%%%%%%%%%%%%%%%%%%%%%%%%%%%%%%%%%%
% set up packages
%%%%%%%%%%%%%%%%%%%%%%%%%%%%%%%%%%%%%%%%%%%%%%%%%%%%
\usepackage{geometry}                
\usepackage{textcomp}                
\usepackage{amsmath}                
\usepackage{graphicx}                
\usepackage{amssymb}                
\usepackage{fancyhdr}                
\usepackage{subcaption}                
\usepackage{bm}                
\usepackage{lineno}

\usepackage[superscript,noadjust]{cite} % puts dash in citations to abbreviate
\usepackage [autostyle, english = american]{csquotes} % sets US-style quotes

\usepackage{etoolbox} % block quotes

\usepackage{float}
\usepackage{color}

\usepackage{pgf}
\usepackage{tikz}
\usepackage{eqnarray}

\usepackage{listings} % code blocks
\usepackage{setspace}

\usepackage{lscape}

\usepackage{natbib}
\bibliographystyle{abbrvnat}
\setcitestyle{authoryear,open={(},close={)}}

%%%%%%%%%%%%%%%%%%%%%%%%%%%%%%%%%%%%%%%%%%%%%%%%%%%%
% call packages
%%%%%%%%%%%%%%%%%%%%%%%%%%%%%%%%%%%%%%%%%%%%%%%%%%%%	
\geometry{letterpaper, marginparwidth=60pt} % sets up geometry              		
\linenumbers % adds line numbers 
\MakeOuterQuote{"} % sets quote style
\doublespacing % setspace

%%%%%%%%%%%%%%%%%%%%%%%%%%%%%%%%%%%%%%%%%%%%%%%%%%%%
% patches with etoolbox 
%%%%%%%%%%%%%%%%%%%%%%%%%%%%%%%%%%%%%%%%%%%%%%%%%%%%	
% block quotes
\AtBeginEnvironment{quote}{\small}

% linenumbers
\makeatletter
\patchcmd{\@startsection}{\@ifstar}{\nolinenumbers\@ifstar}{}{}
\patchcmd{\@xsect}{\ignorespaces}{\linenumbers\ignorespaces}{}{}
\makeatother

%%%%%%%%%%%%%%%%%%%%%%%%%%%%%%%%%%%%%%%%%%%%%%%%%%%%
% tikzlibrary modifications
%%%%%%%%%%%%%%%%%%%%%%%%%%%%%%%%%%%%%%%%%%%%%%%%%%%%	
\usetikzlibrary{fit}
\usetikzlibrary{positioning}
\usetikzlibrary{arrows}
\usetikzlibrary{automata}

%%%%%%%%%%%%%%%%%%%%%%%%%%%%%%%%%%%%%%%%%%%%%%%%%%%%
% page formatting; exact 1 in margins
%%%%%%%%%%%%%%%%%%%%%%%%%%%%%%%%%%%%%%%%%%%%%%%%%%%%
\pagestyle{plain}                                                     

\setlength{\textwidth}{6.5in}    
\setlength{\oddsidemargin}{0in}
\setlength{\evensidemargin}{0in}
\setlength{\textheight}{8.5in}
\setlength{\topmargin}{0in}
\setlength{\headheight}{0in}
\setlength{\headsep}{0in}
\setlength{\footskip}{.5in}

%%%%%%%%%%%%%%%%%%%%%%%%%%%%%%%%%%%%%%%%%%%%%%%%%%%%
% defining code blocks using listings package
%%%%%%%%%%%%%%%%%%%%%%%%%%%%%%%%%%%%%%%%%%%%%%%%%%%%

\definecolor{dkgreen}{rgb}{0,0.6,0}
\definecolor{gray}{rgb}{0.5,0.5,0.5}
\definecolor{mauve}{rgb}{0.58,0,0.82}

\lstset{frame=tb,
  language=R,
  aboveskip=3mm,
  belowskip=3mm,
  showstringspaces=false,
  columns=flexible,
  basicstyle={\small\ttfamily},
  numbers=none,
  numberstyle=\tiny\color{gray},
 % keywordstyle=\color{blue},
  commentstyle=\color{dkgreen},
  stringstyle=\color{mauve},
  breaklines=true,
  breakatwhitespace=true,
  tabsize=3,
  otherkeywords={0,1,2,3,4,5,6,7,8,9},
  deletekeywords={data,frame,length,as,character,dunif,ps},
}

%%%%%%%%%%%%%%%%%%%%%%%%%%%%%%%%%%%%%%%%%%%%%%%%%%%%
%%%%%%%%%%%%%%%%%%%%%%%%%%%%%%%%%%%%%%%%%%%%%%%%%%%%
% begin document
%%%%%%%%%%%%%%%%%%%%%%%%%%%%%%%%%%%%%%%%%%%%%%%%%%%%
%%%%%%%%%%%%%%%%%%%%%%%%%%%%%%%%%%%%%%%%%%%%%%%%%%%%

\begin{document}

\bibliographystyle{plainnat} 

\section*{Appendix X}

I will use this appendix to compare different models for a success/trial dataset to demonstrate shrinkage due to partial pooling.

\subsection*{...}

\subsection*{Maximum likelihood estimate}

Below, I'm including a snippet of MATLAB code that I think has been used to calculate seedling survival to fruiting ($\sigma$) in the past. The code generates estimates for $\sigma$ for each population in each year; I do not believe that the code snippet that is used to calculate \verb|sig2| is used elsewhere in the MATLAB program. 

\begin{lstlisting}
% ESTIMATE ABOVE-GROUND VITAL RATE:
%   sigma, SURVIVAL GERM>FRUITING

% get data on survival germ>fruiting
dfname=[folder 'Survivorship & Fecundity_06-11vers2.xlsx'];
[numdata,txtdata]=xlsread(dfname);

% VARIABLE
% 1     easting
% 2     northing
% 3     site
% 4     transect
% 5     position
% 6     seedling#_1/06
% 7     flow#_6/06
% 8     fruit#_6/06
% 9     seedling#_1/07
% 10    flow#_6/07
% 11    fruit#_6/07
% 12    fruit/pl_6/07
% 13    fl>sdl_07
% 14    seedling#_1/08
% 15    flow#_6/08
% 16    fruit#_6.08
% 17    fruit/pl_6.08
% 18    fl>sdl_08
% 19    seedling#_1/09
% 20    flow#-5/09
% 21    fruit#-6/09
% 22    fruit/pl-6.09
% 23    fl>sdl_09
% 24    seedling#_1/10
% 25    fruitpl#-6/10
% 26    fruit/pl-6.10
% 27    fruitpl#>sdl_10
% 28    seedling#_2/11
% 29    fruitpl#-6/11
% 30    (fruit+fl)/pl-6.11
% 31    fruitpl#>sdl_11
% 32    Notes

Site=txtdata(:,3);
Site(1)=[];
Sdl=[numdata(:,6) numdata(:,9) numdata(:,14) numdata(:,19) numdata(:,24) numdata(:,28)]; % no. of seedlings
Frt=[numdata(:,8) numdata(:,11) numdata(:,16) numdata(:,21) numdata(:,25) numdata(:,29)]; % no. of fruiting plants

NumSdl=zeros(numpops,numyrs);
NumFrt=NumSdl;
nsig=NumSdl;
sigma=NumSdl;
sig2=NumSdl;
MeanSdl=NumSdl;
MeanFrt=NumSdl;
MeanSdlSE=NumSdl;
MeanFrtSE=NumSdl;
s0s1g1=NumSdl; % stores raw estimates of products of 3 vrs from plot data
s0s1g1_bag=NumSdl; 
sigest=NumSdl; 

for p=1:numpops

    PopNow=pops(p);
    
    for y=1:numyrs      
        % only use plots ("positions") with both sdl>0 and frting plt counts
        I=find(strcmp(Site,PopNow) & Sdl(:,y)>0 & ~isnan(Frt(:,y))); 
        nsig(p,y)=length(I);
        x1=sum(Sdl(I,y));
        NumSdl(p,y)=x1;
        x2=sum(Frt(I,y));
        NumFrt(p,y)=x2;
        sigma(p,y)=min(x2/x1,1); % surv. germ>fruiting not allowed to exceed 1
        
        % compute sigma using plot means - doesn't require a plot to have both sdl and fruiting plant counts
        i1=find(strcmp(Site,PopNow) & ~isnan(Sdl(:,y)));
        i2=find(strcmp(Site,PopNow) & ~isnan(Frt(:,y)));
        MeanSdl(p,y)=mean(Sdl(i1,y));
        MeanSdlSE(p,y)=std(Sdl(i1,y))/sqrt(length(i1));
        MeanFrt(p,y)=mean(Frt(i2,y));
        MeanFrtSE(p,y)=std(Frt(i2,y))/sqrt(length(i2));
        sig2(p,y)=MeanFrt(p,y)/MeanSdl(p,y);
        
    end
    
end

\end{lstlisting}

For a single observation, the likelihood that we observe $y$ fruiting plants in a plot if $n$ seedlings were present in the plot can be written as a function of the probability of seedling survival $p$ as $[y|p,n] = \binom{n}{y}p^y(1-p)^{n-y}$. For a set of $N$ observations, each with a number of seedlings $n_i$ and a number of fruiting plants $y_i$ in the $i$th observation, then we can write the likelihood as
%
\begin{align}
  \begin{split}
\mathcal{L} = [\bm{y}|p,\bm{n}]  = \prod_{i=1}^N \binom{n_i}{y_i}p^y_i(1-p)^{n_i-y_i}.
  \end{split}
\end{align}

We can use the likelihood to obtain a maximum likelihood estimate (by minimizing the negative log-likelihood). The maximum likelihood estimate $\hat{p}$ is the overall proportion of seedlings that survive to become fruiting plants, summing all the observations. 

\subsection*{Binomial model with complete pooling and a beta prior}

Next, we'll consider adding a prior to our model and writing this as a Bayesian model. We use a beta distribution for the prior, because the beta is a conjugate prior for a binomial distribution; in other words this choice of prior matches the likelihood in a way that the posterior has the same distribution as the prior (cf. Bolker p 177). A beta distribution with shape parameters $\alpha=\beta=1$ corresponds to noninformative prior. For a set of $N$ observations, each with a number of seedlings $n_i$ and a number of fruiting plants $y_i$ in the $i$th observation, we can write the joint posterior as
%
\begin{align}
  \begin{split}
[\bm{y}|p,\bm{n}]  = \prod_{i=1}^N \mathrm{binomial}(n_i,p) \mathrm{beta} (  p | 1 , 1 ).
  \end{split}
\end{align}

This corresponds to a model in which there is \textit{complete pooling}, with a single probability $p$ representing the probability of seedling survival to fruiting for the all trials. The opposite extreme would be a model with \textit{no pooling}, in which each trial $i$ has its own probability of seedling survival to fruiting $p_i$. 
%
\begin{align}
  \begin{split}
[\bm{y}|\bm{p},\bm{n}]  = \prod_{i=1}^N \mathrm{binomial}(n_i,p_i) \mathrm{beta} (  p_i | 1 , 1 ).
  \end{split}
\end{align}

This corresponds to a model in which there is \textit{complete pooling}, with a single probability $p$ representing the probability of seedling survival to fruiting for the all trials. To compare our site- and year-specific MLE fits, it might be best to run the following model:
%
\begin{align}
  \begin{split}
[\bm{y}|\bm{p},\bm{n}]  = \prod_{j=1}^J\prod_{k=1}^K\prod_{i=1}^N \mathrm{binomial}(n_{ijk},p_{jk}) \mathrm{beta} (  p_{jk} | 1 , 1 ).
  \end{split}
\end{align}
Here, we give each site $j$ and year $k$ its own probability of success $p_{jk}$. Effectively, this is a model in which we are completely pooling observations from each site and year. This is pretty similar to what we are doing with the maximum likelihood estimates when we sum across all the plots at a site in a given year and calculate the proportion of seedlings that survive to become fruiting plants. One difference between the two approaches is that with the Bayesian model we do account for the number of trials and counts; the data from one plot with a single seedling compromises with the prior to give us posterior probability of success. For the most part, though, these estimates are pretty similar (see that most points fall on a 1:1 line).

\subsection*{binomial model with partial pooling and a beta prior}

Next, we'll consider adding pooling to our model. We do this by putting hyperpriors on the parameters for the beta distribution. We are interested in the mean at each site. We'll use the parameterization in Kruschke:
%
\begin{align}
  \begin{split}
[\bm{p},\omega,\kappa|\bm{y},\bm{n}]  = & \prod_{j=1}^J\prod_{i=1}^N \mathrm{binomial}(n_{ij},p_{j}) 
    \\ & \times \mathrm{beta} (  p_{j} | \omega(\kappa-2) +1 , (1-\omega) (\kappa -2) + 1) 
    \\ & \times \mathrm{beta} ( \omega | 1, 1) \mathrm{gamma} ( \kappa | .01, .01)  .
  \end{split}
\end{align}

\subsection*{Multi-level binomial model with partial pooling and a beta prior}

Next, we'll consider pooling across all sites and years in our dataset. We do this by putting hyperpriors on the parameters for the beta distribution. We are interested in the mean at each site. We'll use the parameterization in Kruschke:
%
\begin{align}
  \begin{split}
[\bm{p},\omega,\kappa|\bm{y},\bm{n}]  = & \prod_{j=1}^J \prod_{k=1}^K \prod_{i=1}^N \mathrm{binomial}(n_{ijk},p_{jk}) 
    \\ & \times \mathrm{beta} (  p_{jk} | \omega(\kappa-2) +1 , (1-\omega) (\kappa -2) + 1) 
    \\ & \times \mathrm{beta} ( \omega | 1, 1) \mathrm{gamma} ( \kappa | .01, .01)  .
  \end{split}
\end{align}



\subsection*{Binomial model with partial pooling and a log-odds parameterization}

logit parameterization 

\subsection*{Comparison}

Goal is to compare the output from each of these models. They should be pretty similar for most of the data because it's well-replicated and the sample sizes are reasonable. The differences that I am interested in pointing out are

- what happens when there are few observations (the prior has influence)
- how well can we estimate the site mean or the year mean from each dataset?
- prediction: which of our datasets does a better job at predicting?

\clearpage
\bibliography{/Users/gregor/Dropbox/bibliography/seeds}

\end{document}