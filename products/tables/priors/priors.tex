\documentclass[12pt, oneside, titlepage]{article}   	% use "amsart" instead of "article" for AMSLaTeX format
\usepackage{geometry}                     
\usepackage{amsmath}                      
\usepackage{amssymb}                       
\usepackage{bm}   
\usepackage{tabularx}   
\usepackage{caption}          
 \captionsetup[table]{labelfont=sc}

\usepackage{booktabs}
\usepackage{xfrac}
\usepackage{graphicx}

\usepackage{rotating}
\usepackage{soul}
\usepackage{hyperref}



%%%%%%%%%%%%%%%%%%%%%%%%%%%%%%%%%%%%%%%%%%%%%%%%%%%%
%%%%%%%%%%%%%%%%%%%%%%%%%%%%%%%%%%%%%%%%%%%%%%%%%%%%
% begin document
%%%%%%%%%%%%%%%%%%%%%%%%%%%%%%%%%%%%%%%%%%%%%%%%%%%%
%%%%%%%%%%%%%%%%%%%%%%%%%%%%%%%%%%%%%%%%%%%%%%%%%%%%

\begin{document}

%%%%%%%%%%%%%%%%%%%%%%%%%%%%%%%%%%
% PRIORS FOR HIGHEST-LEVEL PARAMETERS
%%%%%%%%%%%%%%%%%%%%%%%%%%%%%%%%%%

I used priors motivated by a conceptual understanding of Bayesian inference and implementation. I found it particularly helpful to understand that the amount of information encoded by prior can only be understood in the context of the likelihood. \hl{Seaman et al. 2012} and \hl{Gelman et al. 2017} provide a conceptual formulation of this point, and examples include \hl{Hobbs and Hooten 2015, p. 95-97}, \hl{Wesner and Pomeranz 2020, bioRxiv}, \hl{Gelman et al. 2020 - workflow}, \hl{Gabry et al. 2019. p. 393-394}, \hl{Northrup and Gerber 2018}, \hl{Gelman et al. 2006}, \hl{Gelman et al. 2008}, \hl{Lemoine 2019, A1.2}

I used published literature and models to develop priors. The 'default' flat priors recommended for linear models are not always noninformative for models with different likelihood functions. Because none of my likelihood functions were normal (binomial, negative binomial), I identified priors that were relatively noninformative in the context of the joint likelihood for each model. References for these priors include the following: \hl{Hindle et al. 2019, Table S1}, \hl{Rosenbaum et al. 2019, Table 1}, \hl{Hobbs et al. 2015, Table 3}, \hl{Hindle et al. 2018}, \hl{Smits 2015}. I also followed the guidance in \hl{Lemoine 2019} to use positive, unbounded priors on variances, and to use Cauchy priors for the random-intercepts. 

For priors with hierarchical centering, see \hl{Evans et al. 2010, Appendix A \& Table 3}.

Weakly informative priors were assessed by simulating prior predictive distributions \hl{Gabry et al. 2019, p. 393-394}, \hl{Conn et al. 2018, p. 529-530}, \hl{Hobbs and Hooten 2015, p. 85}. This step helped confirm that the chosen joint likelihood (deterministic model, stochastic model) generated data within the observed range. This is similar in logic to the approach taken by \hl{Evans et al. 2010, Methods: Estimating vital rates from demographic data: Priors} except in that case the authors compared their observed means to those generated by their priors.

% BINOMIAL LIKELIHOOD
The parameters for population means ($\mu_{0,ik}^\mathrm{germ}$) in models with a binomial likelihood were given $N(0,1)$ priors [ref: $\beta_0^g, \beta_0^s$, \hl{Evans et al. 2010, Table 3}]. \hl{Evans et al. 2010, Table 3} use a $\mathrm{Uniform}(0,2)$ prior on the variance components in models with a binomial likelihood [ref: $\sigma_{yr}^g, \beta_{yr}^s, \beta_{pop}^s$, \hl{Evans et al. 2010, Table 3}]. I used truncated, normally distributed priors for the parameters for standard deviation for population and population-and-year levels of the hierarchy, in models with a binomial likelihood. For the germination models, I used a $N(0,1)^+$ prior. This is a weakly informative prior [ref: $\sigma_{\dots}$, \hl{Rosenbaum et al. 2019, Table 1}]. See: can I call this a half-normal? 

For the survival models, I used a normal for the rate parameter (should this be a truncated normal?). [ref: $\phi_{\dots}$, \hl{Shriver et al. preprint, Table S1}]. I used truncated, normally distributed priors for the parameters for standard deviation for population and population-and-year levels of the hierarchy. \hl{Smits 2015} places a half-Cauchy prior on the variance term of the random intercept but when I evaluated this prior in the context of the binomial likelihood l found it to have too thick of a tail.

We construct a model to relate all the sources of data (cf. \hl{Metcalf et al. 2009, 2.8: Prior specification})

These priors were also used for the seedling survival to fruiting models, which also have a binomial likelihood. I initially used a $N(0,1000)$ distribution as a prior for the population mean, and a $\mathrm{Uniform}(0,1.5)$ distribution as a prior on the parameters for standard deviation for population and population-and-year levels of the hierarchy. The distribution on the population mean was motivated by my understanding that it was a noninformative, vague prior. The distribution on the standard deviations were based on literature (e.g.  \href{www.mbr-pwrc.usgs.gov/pubanalysis/keryroylebook/R_BUGS_code_AHM_Vol_1_20170519.R}{AHM};\hl{Okamoto et al. 2016, Table B1, among-site variance}. I chose to update my priors for this model after developing a better understanding of how to use prior predictive checks to assess whether a prior was reasonable.

The models for fruits per plant and seeds per fruit use a negative binomial likelihood. For these models, I initially used a $N(0,1000)$ distribution as a prior for the population mean, and a $\mathrm{Uniform}(0,1.5)$ distribution as a prior on the parameters for standard deviation for population and population-and-year levels of the hierarchy. See above paragraph for explanation. An alternative prior for the population mean is $N(0,5)$ [ref: $\beta_3$, \hl{Shriver et al. 2019, Table S1}], which is on the weakly-informative side. An alternative for the standard deviation is a half-Cauchy (0,1) [ref: $\tau$, \hl{Shriver et al. 2019, Table S1}]

 The negative binomial likelihood also takes a dispersion parameter, $\kappa$, that I gave a $\mathrm{gamma}(0.001,0.001)$ prior distribution [ref: \hl{Hobbs and Hooten 2015, p. 253}]. An alternative prior for the dispersion parameter would be a half-Cauchy(0,5) [ref: $\kappa$, \hl{Shriver et al. 2019, Table S1}]. Both these options are positive, unbounded. 

Schultz et al. (2017) use a non-centered parameterization but use $N(0,10)$ priors on means, half-Cauchy(0,2.5) on the standard deviations for random intercepts, and a IG(0.001,0.001) on the overall standard deviation. [See Appendix S1 for the growth model]


Makinen and Vanhatalo 2017 also include details on priors for NB model, as well as for the inverse scale parameter

Next steps: identify reference for the parameter in the beta distribution. Identify reference for parameters in the rate parameter for the exponential survivorship function. Identify references for the parameters in the hierarchy of the joint likelihoods with negative binomial likelihood. 

CONSIDER: change population means to Cauchy because I *do* have small sample sizes sometimes; 

Saturday:

1. write NB model with the original and revised priors, visualize prior predictive distributions
2. Write binomial model with original and revised priors, visualize prior predictive distributions
3. write binomial model with exponential decay process, visualize prior predictive distributions

Notation in all parameters is for site $i$, year/round $j$, age $k$. Superscript plus indicates a distribution truncated to be positive

\footnotesize

\begin{center}
\captionof{table}{ Description of parameters for prior distributions at the highest level (hyperparameters). } \label{tab:title2} 
 \begin{tabularx}{\linewidth}{l l l l} 
 \hline
 \hline
\multicolumn{1}{ l }{ Parameter } & 
\multicolumn{1}{ c }{ Description } &
\multicolumn{1}{ c }{ Distribution } &
\multicolumn{1}{ c }{ Type } \\
 \hline
 
  %%%% SEED BURIALS
   \multicolumn{4}{ l }{ \sc{Seed bag burial experiments}} \\

 %% Population-level means
  $\mu_{0,ik}^\mathrm{germ}$   & Population mean germination   & $N(0, 1)$ & Weakly informative \\ 
  $\sigma_{0,ik}^\mathrm{germ}$   & Population S.D. of germination   & $N(0, 1)^+$ & Weakly informative \\ 
  $\sigma_{0,ijk}^\mathrm{germ}$   & Population and year S.D. of germination   & $N(0, 1)^+$ & Weakly informative \\ 
  
  $\mu_{0,i}^\mathrm{decay}$   & Population mean \hl{rate parameter}  & $N(0, 1)$ & Weakly informative \\ 
  $\sigma_{0,i}^\mathrm{decay}$   & Population S.D. of \hl{rate parameter}   & $N(0, 4)^+$ & Weakly informative \\ 
  $\sigma_{0,ij}^\mathrm{decay}$   & Population and year S.D. of \hl{rate parameter}  & $N(0, 4)^+$ & Weakly informative \\ 
  
    $\beta_{ij}^\mathrm{decay}$   & Population and year shape parameter of beta & $N(0, 1000)^+$ & Vague \\ 

% $\mu_0^1$   & Population mean intact to January 1 & normal(0, 1000) \\ 
% $\mu_0^3$   & Population mean intact to October 1 & normal(0, 1000) \\ 
% $\mu_0^4$   & Population mean intact to January 2 & normal(0, 1000) \\ 
 
 %% Population-level SDs
% $\sigma_0^1$   & Population S.D. of intact to January 1  & half-normal(0, 3$\sfrac{1}{3}$)  \\ 
% $\sigma_0^2$   & Population S.D. of germination in January 1 & half-normal(0, 3$\sfrac{1}{3}$)  \\ 
% $\sigma_0^3$   & Population S.D. of intact to October 1 & half-normal(0, 3$\sfrac{1}{3}$)  \\ 
% $\sigma_0^4$   & Population S.D. of intact to January 2 & half-normal(0, 3$\sfrac{1}{3}$)  \\ 

 %% Population and year-level SDs
% $\sigma^1$   & Population and year S.D. of intact to January 1 & half-normal(0, 3$\sfrac{1}{3}$)  \\ 
% $\sigma^2$   & Population and year S.D. of germination in January 1 & half-normal(0, 3$\sfrac{1}{3}$)  \\ 
% $\sigma^3$   & Population and year S.D. of intact to October 1 & half-normal(0, 3$\sfrac{1}{3}$)  \\ 
% $\sigma^4$   & Population and year S.D. of intact to January 2  & half-normal(0, 3$\sfrac{1}{3}$)  \\ 
 
  %%%% SEEDLING SURVIVAL TO FRUITING
 \multicolumn{4}{ l }{ \sc{ Seedling survival to fruiting }} \\

 %% Population-level means
 $\mu_{0,i}^{\mathrm{surv}}$   & Population mean seedling survival & $N(0, 1)$ & Weakly informative \\ 
 
 %% Population-level SDs
 $\sigma_{0,ij}^{\mathrm{surv}}$   & Population S.D. of seedling survival   & $N(0, 1)^+$ & Weakly informative  \\ 

 %% Population and year-level SDs
 $\sigma_{0,ij}^{\mathrm{surv}}$   & Population and year S.D. of seedling survival  & $N(0, 1)^+$ & Weakly informative \\ 
 
   %%%% NEGATIVE BINOMIAL LIKELIHOOD
    \multicolumn{4}{ l }{ \sc{ Negative binomial likelihood: option 1 }} \\

 %% Population-level means
 $\mu_0^{\mathrm{seeds}}$   & Population mean fruits per plant & normal(0, 1000) \\ 
 
 %% Population-level SDs
 $\sigma_0^{\mathrm{seeds}}$   & Population S.D. of fruits per plant & uniform(0, 1.5)  \\ 

 %% Population and year-level SDs
 $\kappa$   & Dispersion parameter & gamma(0.001, 0.001) & Hobbs and Hooten  \\ 
 
    %%%% NEGATIVE BINOMIAL LIKELIHOOD
     \multicolumn{4}{ l }{ \sc{ Negative binomial likelihood: option 2 }} \\
     
  $\mu_0^{\mathrm{seeds}}$   & Population mean & $N(0,5)$ & Shriver 2019 \\ 
 $\sigma_0^{\mathrm{seeds}}$   & Population S.D. & half-Cauchy(0,1) & Shriver 2019  \\ 
 $\kappa$   & Dispersion parameter & half-Cauchy(0,5) & Shriver 2019   \\ 

 
  \hline
\end{tabularx}
\end{center}

\newpage
\end{document}